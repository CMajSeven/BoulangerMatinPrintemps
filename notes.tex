\documentclass[twoside]{article}
\usepackage[a3paper, margin=16mm, includehead]{geometry}
\usepackage{changepage}
\usepackage{enumitem}
\usepackage{fancyhdr}
\usepackage{fontspec}
\usepackage{hanging}
\usepackage{multicol}
\usepackage{wasysym}
\setmainfont[
 BoldFont={[ACADEMICO-BOLD.otf]}, 
 ItalicFont={[ACADEMICO-ITALIC.otf]},
 BoldItalicFont={[ACADEMICO-BOLDITALIC.otf]}
 ]{[ACADEMICO-REGULAR.otf]}
\setlength{\columnsep}{1cm}

\pagestyle{fancy}
\renewcommand{\headrulewidth}{0pt}
\fancyhf{}%clear all headers and footers
\fancyhead[LE]{\fontsize{8pt}{10pt}\selectfont \slshape\rightmark}
\fancyhead[RO]{\fontsize{8pt}{10pt}\selectfont \slshape\leftmark}
\fancyhead[LE,RO]{\thepage}
\setcounter{page}{28}

\begin{document}
\begin{center}
\underline{\huge{Editorial Commentary}}
\end{center}

This edition is based on manuscripts of the scores and parts by a copyist from Lili Boulanger's publisher Durand. There are numerous errors and inconsistencies, listed below. Those of significance are italicized, or bolded and italicized, and are inserted as footnotes in the score. I cannot guarantee that this is a complete list. Some editorial decisions have been made based on the duet version for violin or flute and piano, but due to the many differences (including in harmony) between the orchestral and duet versions, the duet version cannot be considered a point a truth.

\begin{hangparas}{15pt}{1}
\bigbreak
FS = Full Score
\begin{multicols}{2}

\underline{General}

\textit{1: Tempo marking in FS has }$\quarternote=126$\textit{, written in a different hand. Duet has tempo marking of }$\quarternote\ =144$\textit{.}

200--203: Staff level crescendos have various start and stop points here. System level crescendo over all strings. Making all crescendos extend to 203

\underline{Piccolo}

22--28: Differing hairpin lengths between part and FS

28: Part has ♯ on G, FS has ♯ on A. Part matches duet.

46: FS has p far above the staff. Part has f. Adding ``rythmé''

135--137, 178--180: Differing articulation between part and FS, matching celesta and flutes. Dynamics only in part

178: FS has F♯ instead of F (ref. violins). Slur break before 179 in part

180: FS has F♯, part has G♯. No one else has F♯ so G♯ is probably correct.

197: Articulation only in part. Removing marcato (leaving staccato) for consistency

202: ff only in part. Moving ff to 203 and adding crescendo

\underline{Flutes}

\textit{3: FS is not marked with any player indication. Part has only player 1 marked solo, and ``col 1'' written over rests for player 2.}

3--4: Staccatos only in part

6: FS has 16th, 16th, 8th. Part has 8th, 16th, 16th. Part matches duet.

10: FS has dotted 8th, 16th. Part has 8th, 16th rest, 16th. Dynamics only in part

11: FS has A4. Part has E4. Part matches duet.

12: Flute part only has flute 1 playing. Ambiguous in FS. Adding mf from clarinet part

13, 15: Staccatos only in part, matching duet version

12--14: Various crescendo lengths in flute and clarinet parts

20: FS flute 1 has G instead of F♯.

23: FS flute 1 has E♯ as first note instead of F♯.

31: Beaming discrepancies between part and score. Score matches duet.

40: Part has (badly drawn) decrescendo.

41--44: Slurs only in part

44--45: D♯ trill continues into 45 in part but not FS. No slurs in FS

45: Part has stray ff under rest before 46.

46--47: Normalizing dynamic to ``f rythmé''. FS has f, part has ff and f written.

55: Decrescendo and accent only in part

59: Flute 2 decrescendo only in full score, accent only in part

68--71: Dynamics only in part

76--78: Flute 2 crescendo only in FS

\textit{88--89: Flute 2 in part is blank (``col 1'' slashes stop). It is unclear whether it is still unison with flute 1. (In any case does not match FS.)}

100: FS and part don't have p on 2nd 8th note like the previous measures. Adding p in

104: Slur in FS, staccato in part

112: Decrescendo missing in part

132: Slur extends over beat 1 in FS, only to first note in part.

133--137: Articulation differences, particularly part has beat 3 8th note as staccato 16th (still under slur).

155: FS flute 1 has C tied to C. Part has D slurring to C. D is 3/4 in C space so could be error.

156--157: Crescendo only in part

172: Inserting ff

174: Missing slur in part

184: f only in part (potentially ff but second f is faded). Normalizing to ff from other woodwind parts

\textit{188: Parts have C♯6 and F♯6, which would match piano version, crossed out with G♯5 and D♯6, matching FS.}

192: Crescendo only in part

201--203: Dynamics only in part. Original crescendo extends 3 beats.

\underline{Oboes}

7, 22: Part has ppp. FS has pp.

10: Oboe 2 tie missing in FS

19: FS has E on last 16th instead of C (ref. flute).

25: Adding crescendo from flutes

28: FS missing ♭ on A♭

40, 45: Crescendo only in part

45--46: D♯ tied in part but not FS

46--47: Normalizing dynamic to ``f rythmé''

48: Oboe 1 graces are A5,G5 in FS and A5,B5 in part. Oboe 2 graces are A4,B4 in FS and E5,F♯5 in part. ff only in part, seems poorly balanced with other parts

49.3: FS has C5+G5, part has G5+C6.

50--51: Slurs connect full runs in part but not FS.

51: In FS, oboe 2 run is from A to E, instead of octave below oboe 1.

60: mf only in part

61: Decrescendo only in part. Beat 4 quarter note in FS, 8th note in part

76--81: Oboe 1 slurs extend over last note on line but aren't drawn past line break in both FS and part (at different places). Continuing slur until 81.

82: Part has f with m added in front. No dynamics in FS

83--84: Slur extends to F♭ in FS but not in part. Oboe 2 has different slur divisions.

\textbf{\textit{86--89: This section is crossed out in the part. In part, oboes play only an 8th note F4 at ppp with decrescendo in previous measure.}}

87: Trill should perhaps be ♯ to match flute. No part has an E♮. ♮ is in both FS and part.

92: pp and decrescendo only in part, messily drawn in different hand. Copying dynamics from flute part

98: Beam split for consistency. ``Solo'' only in part. FS missing ♭ on G♭

112: Slur only in part

138--139: Slur breaks with measure in FS. Slur breaks after measure in part.

160: p only in FS

165: Accents only in part

166: sf only in part

172: ff only in part

181: FS has ♭ on E.

\textit{188: See note for flutes}

192: Adding crescendo from other woodwinds

200: Adding p

200--203: Crescendo originally stops at 202.2.

203: Adding ff

205--206: Crescendo only in part

\underline{English Horn}

25: No slur in part beat 1. Slur starts from last 8th note in part, from dotted quarter in FS.

27.3: Part has C♯-D. FS has D-E♭. Piano version has concert G here.

43: FS missing ♯ on C♯ (ref. concert F♯ in other parts)

44: Crescendo only in part

49: Crescendo only in FS

50: Accent only in FS

52.3: E should probably have ♮

58: Adding mf

75: Slur starts on A in part. Crescendo only in part

80--82: Dynamics only in part

83: Accent missing in part

84--85: Decrescendo only in part, drawn in different hand

\textbf{\textit{86--89: Section only in FS}}

161--162: Normalizing dynamics between English horn, bass clarinet, and bassoons

162--163: Part has G♯s. FS has Gs. Piano version has B♯ so FS likely correct.

168, 170: f only in part

172: Inserting ff

181: Decrescendo only in part

184: Inserting ff

188: FS missing ♮ on F

190: Crescendo only in part

192: Crescendo only in part

194: Inserting mf

201--203: Adding crescendo to ff

202.3: ♭ on E only in part. E♮ matches horns. E♭ matches others.

\underline{Clarinets}

12: mf solo, only in part. Not including ``solo'' as it is with flute

13, 15: Adding staccatos based on duet version and flute part

14: Part has E♯ instead of E♮.

18: ``Solo'' only in part

20--21: Crescendo only in part

25: Adding crescendo from flutes. pp sub. at 26 makes little sense without it.

30--31: Splitting beams for consistency with other parts

31.1: FS has A♭+B♭ instead of F+G.

34: Crescendo only in FS

45: Dynamics only in part

46: Part has ``ff très rythmé''. FS has ``f''. Normalizing dynamic to ``f rythmé''

47: Slur extends to 8th note beat 3 in part.

48--49.1: FS has no grace notes or beat 1 for clarinet 2. ff only in part, seems poorly balanced with other parts

49: Adding sf based on other woodwinds

50: Clarinet 2 missing ♮ on A

50--51: Slurs connect full runs in part but not FS.

52--53: Slur extends to 8th note beat 1 of 53 in part.

54: Part has mf. FS has p.

54--56: Slur only in part

56: Crescendo only in part, beats 2 and 3. Moving to 1 to match others

56--58: Slur only extends to end of 57 in part. FS is consistent with bassoon part.

59: Respelling C♯ to Db

60: Adding mf to clarinet 2

76: Dynamic only in part

82: Part has mf written over f. FS has f.

82.1.5: Unison E in FS, D+E in part

82--84: Inserting slurs based on oboe, missing between first 2 notes and second phrase of clarinet 2

85: FS clarinet 1 last 8th note has E♭, likely supposed to be D.

\textbf{\textit{86--89: This section is crossed out in the part. In part, both clarinets play an 8th note G♯5 at pp with decrescendo and different notes in previous measure.}}

\textbf{\textit{90--94: This section is only in the part.}}

95: Part has B♯ instead of B♮.

106: Marcato and staccato only in part

111: sf only in FS. Adding p on second 8th based on previous measures

\textit{129--137: This section is only in the part. It is the same as the flute solo here.}

141: Decrescendo only in FS

144: Part has pp. FS has p.

160: Copying sfp from oboes. Part and FS only have sf.

163: Clarinet 1 has E♮ in part, FS has no accidental. Piano part in duet has concert B♯s. Copying crescendo from oboes

166: Trill accidentals only in part. Crescendo after(?) grace notes only in FS

167: f only in part

168--172: Dynamics only in part. Inserting ff at 172

174: f only in part

181: FS missing ♭ on C♭ in last beat of Clarinet 2 (to match G♯ of other parts). Making beaming consistent with other parts. Decrescendo only in part

188: sf in part here likely a misreading from the bass clarinet and bassoons as no other upper woodwinds have sf.

192: Crescendo only in part

196: Part has ff with first f drawn in different hand. FS has f.

200--203: Crescendo originally stops at 202.2.

205: (Redundant) ff only in part

\underline{Bass Clarinet}

Part transposed from A to B♭

29: Slurs missing in part

60--61: Tie only in part

77: pp only in part, written in different hand

79--80: Tie only in part

82: ``léger'' only in part. Tie into first note, and slur starts at second note in part. No tie, slur over whole measure in FS

143: Bassoon part copied onto bass clarinet staff

161--162: Normalizing dynamics between English horn, bass clarinet, and bassoons

167.3: Part has A-B. FS has A♯-B. To match bassoon and have consistent half-steps, FS is correct. Crescendo in FS only, matching other parts

176: ff only in part

183: Adding sf based on other instruments

187: Crescendo only in part

189: FS has E♮s. Part has E♭s. E♭s match other instruments.

190: Whole measure staccato only in part

196: f in FS, ff in part

201--203: Crescendo originally stops at 202.1.

205: ff only in part, moving to 203 to align with other parts

\underline{Bassoons}

19--20: Different crescendo position between FS and part

25: Slur in part doesn't include first note.

35.2: A3 in part. A2 in FS.

48--49: Crescendo extends through half note in part, stops at 49 in FS.

50: FS has bass clef; part has tenor clef with notes on same lines, though lacking flats at 51.3. Part has bass clef change at 52. Based on cellos, clef change to bass clef should be at 52, and flats in part are correct.

50: Accent beat 1, f beat 2 only in part

56: mf only in part, seems poorly balanced with other parts. Changing to p

56--58: No slur in FS

58--59: mf only in part. Part has crescendo in 58 then decrescendo in 59. FS has only a \textit{crescendo} in measure 59.

\textbf{\textit{68--76: Part is different from FS, placed in ossia.}}

76--81: No slur in FS bassoon 2

79: FS missing ♯ on C♯ bassoon 1

82: f only in part

\textit{82--85: Bassoon 1 part only in FS}

83--84: Tie bassoon 2 only in part

90: Bassoon 2 has A♭ in part, A in FS. Part correct based on duet version.

90--93: Dynamics only in part

140, 142: pp only in part

142.2: Should have C instead of C♯, F instead of F♯

\textit{143.2--.3: Section only in part}

158--159: Decrescendo only in FS

161--162: Normalizing dynamics between English horn, bass clarinet, and bassoons

176, 184: ff only in part

188.3: Bassoon 1 in part has B♯, FS has explicit B♭. Minor second probably correct, so bassoon 2 should probably have E♯.

196: Part has ff instead of f, and then f at 198. Normalizing to f from other parts

201--203: Crescendo originally starts at 201.2 and ends at 202.1. Adding ff

\underline{Sarrusophone}

196: Part has both ff and mf written. FS has mf.

202--203: Adding dynamics

\underline{Horns 1 and 2}

16--17: Hairpins only in part

26--28: Ties only in part

42: Crescendo only in part. Horn 2 missing legato in part

40--43: Normalizing dynamics from strings and between horns (cresc.)

44: Horn 1 accent only in part

46: ``léger, rythmé,'' staccato only in part

53: p only in part

\textit{61--64: Part is with mutes, removed after 64. FS is without mutes.}

80--81: Crescendo only in part

86--87: Crescendo only in part

88: FS has D, C♭, B♭; part has F, E♭, D♭. Based on cellos and duet version, part is correct.

\textit{90: FS has half note B♭. Part has quarter note B♭ then dotted quarter note G♭ (like cello part).}

121: ppp only in part

128: Decrescendo only in part

170--171: Crescendo only over 170.2--.3 in FS

172: ff only in part. Changing to f to match horns 3 and 4, and general balance

172--173: No ties in FS

173: Crescendo in FS only

174: Crescendo only in part

180: p only in part

181: Decrescendo only in part

183: Accents only in part

190: mf only in part

191: Adding ff

196: mf is on 2nd 8th in part, at start in FS.

197: Slur covers whole measure in part, starts at 16th notes in FS.

200--203: Crescendo originally stops at 202.

207: A+E in FS and parts should be a whole step up to fit concert E minor chord.

\underline{Horns 3 and 4}

16--17: Hairpins only in part

20: Horn 4 part has sf. FS has mf. Normalizing to sf

36--39: Slur missing in FS

40--41: Crescendo only in part

40--43: Normalizing dynamics from strings (cresc.)

44--45: Crescendo covers measure 44 in part as well.

\textit{62--64: See horns 1 and 2.} p only in part

80--81: Crescendo only in part

84: Dynamics only in part. FS missing ♭ on G♭

\textit{90--91: Horn 3 has B♭, C in FS; A♭, A♮ in part. Both notes found in harmony in duet version}

94--95: Decrescendo only in part

148--152: Slur starts at 149 in FS.

170: FS has C♮. Part has C♯ (♯ drawn over a ♮). C♮ would create same relation with C♯ of horn 2 as the clarinets have.

172--173: No ties in FS

173: Adding crescendo from horns 1 and 2

175: Horn 2 doesn't play in FS.

183: The crescendo is extended from previous measure only in part. Accents only in part

190: p only in part

191: Adding ff. ♯ on C♯ missing in FS

194: f in part, mf in FS. mf would agree with horns 1 and 2.

197: See slur notes horns 1 and 2

200--203: Crescendo originally stops at 202.

205: Horn 4 has B in part, C in FS. Part probably correct to match E in other parts

207: C+E in FS and parts should be a whole step up to fit concert E minor chord

\underline{Trumpets 1 and 2}

46: Adding slur sim. woodwind parts at 48

47: Slur on grace notes only in FS

50: f only in part

52--53: Tie only in part

79--89: Should likely be unmuted. No indications in part or FS. Indication for ``sourdine'' in part and FS at 95

84: Adding decrescendo based on low brass

126--127: Decrescendo only in part

127--128: Tie only in part

169--172: Dynamics only in part

172.1: Part has quarter slurring into beat 2. FS has marcato, no slur. Changing to match woodwinds

176: Adding staccatos on the 8th notes. Trumpet 2 may require B♭ trumpet

183: Crescendo and ff only in part

184.1.5: Trumpet 2 has C♯ in part where FS has 8th rest.

187--188: Tie only in part

190--191: Dynamics only in part

196--200: Trumpet 2 and 3 parts swapped between FS and parts. i.e. in FS, 1 and 2 are a2. In parts, 1 and 3 are à 2. As 3 is optional, parts are more logical.

201--202: p only in FS

201--203: Crescendo originally stops at 202.

\underline{Trumpet 3}

172: Part has ff. Normalizing to f

174--175: Adding crescendo sim. other trumpet parts

191: Adding ff sim. other trumpet parts

196--200: See note for trumpets 1 and 2

201--202: Part has ff with no dynamics in part. Normalizing to match trumpet~1

205--206: Adding tie sim. other trumpets

\break

\underline{Trombones 1 and 2}

84--88: Dynamics only in part

85: Trombone 2 has E in FS, E overwritten with F and ``fa'' written below in part. F probably correct based on duet version and other parts

172--174: Part has no accent, marked p, and all tied. FS has accent, f, and no ties.

182--183: Part has slur and no accents. FS has no slur and accents.

189--190: Adding mf and p to match other parts

191: Trombone 2 ♯ missing on C♯ (ref. cellos)

198: Adding articulation based on previous measures and tuba part

\underline{Trombone 3}

82--83: Part has ppp instead of pp, crescendo extends over 82--83 instead of just 83. Copying dynamics from other trombones. Slur only in part

87--89: Dynamics only in part

174--176: Slur only in part

182--183: Part has slur and no accents. FS has no slur and accents. Crescendo only in part

189: mf only in part

191: Adding ff

198: Adding articulation based on previous measures and tuba part

199--200: Slur only in part. sf only in FS (and other trombone parts)

201: mf in part. p in FS.

\underline{Tuba}

82--83: See trombone 3

87, 89: Adding dynamics from trombone 3

174: Copying dynamics from trombones

190: Crescendo only in part

191: Adding ff

198: Marcato only in part

200: mf only in part, reducing to p

200--203: Crescendo originally starts at 201.2.

\underline{Percussion}

Use of up- and down-stem voices in ``snare drum or castanet'' part is very inconsistent and has unclear meaning. Using only up-stem

1: pp in part, p in FS

22: pp only in part

30--31: Cymbal trill ends at 8th note in FS, goes through in part.

46: Down-stemmed

47.1: Double-stemmed

48: Up-stemmed in FS, down-stemmed in part

49.1,.3: Double-stemmed only in part

50--51: Non--grace notes down-stemmed in FS (divisi with triangle), up-stemmed in part

54: pp only in FS

95, 97: Up-stem in FS, down-stem in part

96: Double-stemmed

95--96: Crescendo and f only in part

97: p only in part

103, 105: Up-stem in FS, down-stem in part

104, 106: Double-stemmed in FS, down-stem in part

108: ppp only in part.

\textit{121: This cymbal hit is at 122 in FS.}

197: f only in FS

204--206: All tied in FS, no ties in part. Cutting tie at 206 to match trill

\underline{Celesta}

1--9: Part has staccatos. FS has legatos (on top voice only). Using FS articulation as it is notated distinctly from woodwinds

\textit{11: Violin II part, viola part and FS, and celesta FS have E♭, while violin II FS, and celesta part (crossed out ♭) have E, which agrees with duet version.}

11--12: Dynamics only in part

82--88: Slurs only in part

85: FS missing ♭ on upper A♭

88: LH has bass clef instead of treble in FS.

90--92: Slurs only in FS

115: FS has D grace note. Part has D♯. Minor second of part probably correct. Trill accidentals only in part

119: FS has G♯ grace note. Part has G$\times$. Minor second of part probably correct. Trill accidentals only in part

\textit{127: The arpeggio starts after an 8th rest in FS, at the beginning of the measure in part.}

129: ``gai'' only in part

\textit{129--136: Only the upper octave is in the part.}

140.1 RH: B in FS marked as B♯, probably an error

\textit{140.3 LH: G in FS. A with ``la'' written in part. A would be consistent with harp phrases. See harp}

\underline{Harp}

15: ``étouffez'' only in part

54: mf only in part

76--79: Written on celesta staves in FS

76.1.5: A♯ in FS, A in part. Probably A based on duet version

80: FS missing accent

79: First 8th note of beat 2 has E♭ in part, G♭ in FS. G♮ after in both, FS probably an error

81: First group has G♯ in FS, probably an error because of later G♮s. G in part. Second group has B3 after G3 in part, which is missing in FS. Last note is F♯6 in FS, D♯6 in part. F♯ probably correct

82: ``main gauche en dehors'' only in part

84: ♭ missing on D♭ in part and FS

84.2: LH has E3 in FS, D♭3 in part. E matches cellos (F♭). E would match previous sequences better as well.

86: Slur split here in FS but not in part

87: FS has D3 instead of F3, as in part and in previous pattern. F3 would match cellos as well.

91: FS missing ♭ on lower A♭

125: LH in FS has E♭s instead of F♭s

133: Legato only in part

134: FS ♯ missing on C♯ LH

138: FS has bass clef, should be treble clef

\textit{140.3 LH: G in FS. A in part. A would be consistent with surrounding measures.}

183: Crescendo only in part

194--195: Crescendo is in 194 only in FS, over both in part

196: f in part, ff in FS

198: ff only in part (FS has ff continuing from 196)

206: Missing glissando, ♯ missing on D♯ in FS

\underline{Violin I}

5--6, 9--10: Dynamics only in FS

11: Slurs stop here in FS, at 12 in part, corresponding with page breaks. Stopping slurs at rehearsal 1 (measure 12)

12: p in FS, pp in part

16--17: Dynamics only in FS

21: FS has F5+G♯5 and F6+G6. Part has F♯5+G5 and F♯6+G6. Part correct based on duet version. Splitting beam to match later section and duet version

26--28: FS has slur/tie only between 26 and 27. Extends to 28 in part

28: FS missing ♭ on A♭

31: sf only in part

36--38: Slurring different, using part slurs

42--46: Part has slurs but also bow markings, removing slurs

52: Decrescendo only in FS

68: ``la moitié'' crossed out in part

\textit{72: Lower divisi doesn't play the D♭ in FS.}

73: pp for upper divisi only in part. Crescendo only in part

73: ♭ on trill only in FS

76: Upper divisi, tailing grace notes missing in part

76: Lower divisi, slur only in part

77: F instead of F♯ in part upper divisi. Crescendo only in part lower divisi

79: B instead of B♭ in part

80--81: Upper divisi, slurs only in part

82--86: Slurring different, using FS slurs

86: dim. only in part

93: pizz. direction missing in part

113: ``p, très léger, gai'' only in part

\textbf{\textit{115: In part, the run upwards run is F♯ G♯ A♯ B♯ C♯ D♯. In FS, it is F♯ G♯ A♯ C♯ G♯\^{} D. Part matches duet version.}} Slurring different

\textbf{\textit{117--120: FS violin II solo is written here and crossed out.}}

123: Part has 8th note. FS has quarter note.

132: Slur only in part

134: FS upper divisi missing ♯ on C♯

135: Crescendo only in part

142: ``mf, de la pointe'' only in part

143: Part continues melody (clarinet part) but is crossed out.

155--158: Slurring different. Neither match duet version. Using part version

159: Slur missing in FS

167: sf only in part

168: p only in FS

179: No slurs, normalizing with other parts

181: Part missing ♯ on C♯

181--182: Slur only extends to first note of 182 in FS. Using part slur to match previous iterations

182--183: Crescendo at 182 in part, at 183 in FS. Tie only in part

184: Part has ``sans lenteur.'' FS has ``sans lourdeur.''

193--195: Crescendo only in FS

196--197: f to sf in FS, ff to ff in part

200: FS has fp, part just has f. Part has crescendo only at 202.

204--205: Crescendo (to ff) only in part

\underline{Violin II}

7: p in part, pp in FS

\textit{11: See celesta}

11: Slurs stop here in FS, at 12 in part, corresponding with page breaks. Stopping slurs at rehearsal 1 (measure 12)

12: p in FS, pp in part

14.3: FS has B4+G5, part has G4+D5. Harmony here (violins + violas) is different from duet version. Part would match general sequence better.

16: FS has F5, part has F♯5. Part matches duet version.

19.2: FS has C♯5, part has A4. Part matches duet version.

24--27: Hairpins only in FS. pp only in part

26--27: Tie only in part

32: ``pp de la pointe'' only in part

35: Crescendo only in part

39: Crescendo only in part

39.3: Part 1 FS has F5, part has F♯5. Part matches duet version.

40: p only in part

46: f only in part

56, 57: FS has crescendo at 57, part at 56. FS seems more logical.

67: Decrescendo only in part

68--71: Slurs in FS extend to last note in these measures. Part, rest of similar phrases in FS, cellos only extend to second last note.

69: Upper staff labeled as solo in FS. Likely mistake as there is no div. en 2 indication afterwards. Part has div. en 2.

71: FS has A♭, part has A♭♭. Based on flute and duet version, should be A♭

78, 79.2--.3: Upper divisi slurs only in part

80--81: Lower divisi slurs only in part

81: FS grace note has E♯, part has E (no accidental).

82--85: Slur over all 4 measures in part, over each measure in FS.

86.1.5: Part missing ♭ on C♭

\textit{88.2: Part has B♭. FS has D♭.}

\textit{90--95: In FS, div. in 2 and only half plays. In part, unis throughout.}

91--92: Decrescendo only in FS (viola shares decrescendo)

94: Decrescendo only in part (viola shares decrescendo). Adding ``pas sec'' from viola part

110: Part has C♭ instead of C.

113: pp only in part.

\textit{117--120: This solo is not in the part. It is in the violin I part.}

117: FS has A instead of G (ref. Vl. I solo and duet version).

\textbf{\textit{119: Sim. Vl. I:115}}

135.2: 2 and 3 have B and A in FS, D and C in part. Part is correct based on duet version.

142.3: FS missing ♮ on D

\textbf{\textit{152--159: Part has pizz. 8th notes. FS has arco syncopated quarter notes.}}

\textbf{\textit{161: Part has 8th notes on G and E♭. FS has 32nd note tremolo between G and E♭.}}

162: Crescendo only in FS. Slur over whole measure to next in FS, only over grace notes in part

162: Accidentals on trills only in part

167: sf only in part

168: Crescendo starts here in part, at 169 in FS.

169--170: Part has sf. FS has mf.

171: Lower divisi has B in FS, B♭ in part. B♭ matches duet version and other parts. sf only in FS

179: Part has legato on E.

182.2: Top divisi has F4 in FS, B3 in part. F is probably an error.

186: FS missing ♯ on C♯

189--190: Slurring different

192--196: No dynamics in FS. Only a crescendo from 192.3 over 193 in part. Adding dynamics from Vl. I

196: Note lasts 1 beat in FS, 1.5 in part.

197: sf only in FS

200: FS has fp. Part has ff.

204--206: Slur split at 205 in FS. Slur extends over whole run in part.

205: Crescendo (to ff) only in part

\underline{Violas}

11+: FS has alto clef instead of treble clef.

\textit{11: See celesta}

11: Slurs stop here in FS, at 12 in part, corresponding with page breaks. Stopping slurs at rehearsal 1 (measure 12)

14: Part has F$\times$s where FS has Fs. Part is correct based on chamber version. Changing to G like violin II

20: f in part. sf in FS. Should probably be mf with sf like other strings

21: Crescendo goes through beat 2 in part.

26: FS has ``gracieux.'' Part has ``généreux.''

27: Decrescendo only in FS

28: pp only in part

32: pp only in part

\textbf{\textit{32--35: Div. en 3 in FS. div en 2 in part. Lower staff of div.2 split}}

46: f in FS, ff in part

47: Lower divisi first tremolo between A and B in FS, between A and C in part. FS matches violin II.

52--53: Decrescendo only over 52.2 in part

54: Slur extends from 53 in FS

61: No slur in FS

64: sf is on the solo pizz. in part, but on the altri trill in FS. f only in part (which is louder than any other part). Changing to sfp like similar violin I section

68--77.1: Alto solo is tacet instead of unis in FS. Part is unis. ``div en 2'' written in FS but does not continue after 68.

77: Transition from solo to div with no indication at 79/82 so it is unclear whether this is div or solo+altri. More likely div.

78: Slur only in FS

88: Lower divisi missing ♭ on C♭ in FS

94--95: Slur over both measures in FS, one measure each in part

\textit{113--127: The double stop chords in part are split in FS (div.~1~$\rightarrow$~div.~2, div.~2~$\rightarrow$~div.~3). Solo at 121 continues from respective version.}

\textit{125--129: Solo is muted in part, unmuted in FS.}

125: Accent and ``un peu en dehors, intense'' only in FS. mf only in part

\textit{129: Solo does not have the E in FS.}

131--138: Dynamics only in FS

137.1: FS has Fs, part has Gs. Part would match sequence from 11.

144: ``jeu ordinaire'' missing in FS

152--155: Slurs missing in FS

154: Crescendo only in FS

162: ♯ on the trill probably missing (i.e. trill to B♯) based on duet

162: FS has G♯ instead of G$\times$.

164--167: Crescendo only in 166 in FS. sf only at 164 and 165 in part

168: Slur and accent only in part

170: Crescendo only in part

179: Part has legato on E.

184--187.1: Legato marks in part

186: FS missing ♯ on C♯

190--191: Adding crescendo based on other strings

192--193: Crescendo only in part

192: Change to alto clef missing in part

196: f in FS, ff in part. Crescendo only in part

200: No dynamics in part

203: ff in FS, sff in part. Only ff in other parts

203.2: p only in part

204: Crescendo (to ff) only in part

\underline{Cellos}

1: Bass clef in FS, tenor clef in part. Both have 2 ledger line note. Should be B (tenor clef) based on chamber version

11: Crescendo only in part covering 2 beats, normalizing to rest of strings

20: Crescendo only in FS

28--31: Dynamics only in part

\textit{32--35: This is unis in part. Only lower divisi plays in FS.}

36: FS missing arco on upper divisi

46: In part, upper divisi has (arco) B3, lower divisi has (pizz) F♯4. In FS, unison B3 grace note to F♯4 unmarked pizz. or arco

52--53: Decrescendo only in FS

64: Solo with tutti only in part. mf only in part. arco missing

66: p only in FS

67: ``lent'' only in part

\textbf{\textit{68--71: Part has held dotted half notes holding first note of pattern. FS has pattern similar to violin II.}}

68: The slur extends over the whole measure only in this measure.

\textbf{\textit{72--89: In part, this solo is unison of all cellos. Lower divisi continues pattern in FS.}}

\textit{78--85: This section of the solo is an octave higher in the part.}

82: Inserting f for non-solo based on other parts

86: mf only in part. ``en dehors'' only in FS

\textit{90--92: In part, solo holds E♭4 for 1 quarter-note and then drops to C♭4 for rest (like horn part). Melody in chamber version holds E♭.}

\textit{90: Lower divisi has E♭4 quarter note + 8th rest in part (continues from solo), rather than continuing pattern.}

90: Last 3 8th notes, part has A♭2 D♭2 A♭2, FS has G2 D2 G2. Part probably correct based on other instruments and chamber version

96: F♭4 in FS, B♭4 in part. Other parts have B♭ so part is probably correct. 

121: ppp only in part

\textit{121--122: Part has 2 pupitres unis. Only 1er pupitre plays in FS. 1er and 2e pupitre parts swapped between FS and part}

129: Part has pp. All other parts have p.

129: FS still marked 2 pupitre but it is dropped later without indication and 2--pupitre col legno would be inaudible.

144: Part has ``simple'' instead of ``souple.''

147: Decrescendo at beginning of measure in FS, at end in part

150--151: Slur only in part

163.2: FS missing ♯ on C♯

166: Accent only in FS. Upbow mark only in part. (162 has upbow mark in both.)

174--175: Trill is split in FS but continues through in part. sf at 175 only in part

188.2: Part has E♭3+B♭3. FS has E♯3+B3. Part matches bassoon.

190: Crescendo only in part

192--195: Dynamics only in part

\textit{196--197: Div. starts at 196 in FS, at 197 in part.}

203.2: p only in part

203: Slur starts from 16ths in part. Unclear in FS. Changing to be consistent with rest of strings

205: Slur continues into 205 only in part.

\underline{Basses}

30: Crescendo only in part

42: mf only in part

44: Crescendo only in part

46: ff only in part

54: pp in FS, ppp in part

58: Tie split here in part

68: ``pp souple'' only in part

\textbf{\textit{68--75: In FS, div. en 2 with upper having similar part to violas and lower playing first beats pizz. In part, this is unison held out dotted half notes of 4 measures each.}}

72: The FS has arco written above the staff, but the upper divisi is already arco. (arco below staff at 86.)

76: cresc. only in part. 

\textbf{\textit{76--93: Only the lower divisi is in part (as unis).}}

82: f only in part

82--85: Slurs missing, inserting

86: Tie split here in part

\textit{91--92: Octave lower in part}

\textit{144--151: The on-beat and off-beat 8ths are split in FS, but are all unis in part.}

147: Decrescendo only in part. Adding crescendo at 146 to balance

150--151: Dynamics only in part

154: Crescendo only in FS

\textit{162--163: In FS, this is arco. In part, this is pizz. without grace notes.}

162: Dynamics only in part

171--175: Dynamics only in part

\textit{176--179: This section is only in part.}

182--183: p and ff only in part

\textit{190--191: This section is only in part.}

193: Dynamics only in part

195: Dynamics only in part

201: An arco indication should probably be here (missing in part and FS).

203: (ff)p and crescendo only in part

203: Slur starts from sixteenths in part and connects to 204. Unclear start in FS, does not continue to 204. Changing to be consistent with rest of strings

\end{multicols}

\end{hangparas}

\end{document}
